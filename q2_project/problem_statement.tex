
\section{Problem}
\subsection{Model Definition}

We take a grid, upon which there are players which wish to traverse to their
destination. We wish to use this model to find the minimum communication
required for each player to arrive at it's destination. We make the following
assumptions:

\begin{enumerate}
	\item Each player has a pre-defined endpoint
	\item The players each wish to minimize the number of edges traversed to
		reach their destination
	\item The graph edges have no weights
	\item There are no obstacles. In other words, each player can traverse
		any edge freely
	\item Each player has full knowledge of the grid
	\item All player start and end points are unique
	\item The players are traversing a grid
	\item There exits a shared random string
	\item All players move 1 edge at a time
	\item There are 2 players
	\item The map is known in advance of the algorithm's design
	\item All players can communicate
	\item No two players may occupy the same node at the same time
	\item Both players take their turn at the same time
\end{enumerate}

Using this model and varying the assumptions we are able to prove some lower
bounds on the communication complexity for this problem. We start by using all
listed assumptions, and find a $\Omaga(1)$ lower bound for this problem

\subsection{Lower bound with all Assumptions}

We start with the following claim:

\begin{claim}
	Both players can reach their endpoints without collisions using
	$\Omega(1)$ bits of communication
\end{claim}

\begin{proof}
	We propose a simple zero communication protocol for both players to
	traverse the grid. Since each player can see the start and endpoints of
	the other player, they can both identify all possible paths which
	minimize the distance traveled. Since the players are attempting to
	traverse only between two points, there will always be a pair of paths
	following the shortest distance which intersects a maximum of once.
	Player collision can then be determined entirety by the intersection
	point. If there are shortest paths with no intersection point, than it
	is trivial for both players to simply follow those paths and not
	intersect. Otherwise, one of two cases emerges. In the first case, all
	possible points of intersection are equidistant from the player start,
	in which case one player (which one can be agreed upon in advance)
	simply holds for a turn, while the other player moves, thus
	guaranteeing that they will not intersect at the same time. If
	non-equidistant points exist, both players can use a shared
	protocol to pick a point of intersection, and follow one of the paths
	which intersects at that point. Since the point is not equidistant, one
	player will reach it before the other, thus guaranteeing they do not
	visit it at the same time.
\end{proof}

The above model and assumption set is clearly capable of modeling communication
over this graph, and these assumptions can be modified to prove more general
bounds. For instance, if the graph is not known in advance, it is not
necessarily possible to create a deterministic protocol which will allow both
players to identify an intersection point
