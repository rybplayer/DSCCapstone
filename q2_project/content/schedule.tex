We allocate two weeks to reading to ensure we understand the subdomain and are not needlessly duplicating results.

\begin{table}[h!]
    \centering
    \renewcommand{\arraystretch}{1.3}
    \begin{tabularx}{\textwidth}{|c|X|X|X|X|X|}
    \hline
    \textbf{Week} & \textbf{Focus} & \textbf{Ciro} & \textbf{Darren} & \textbf{Ivy} & \textbf{Ryan} \\
    \hline
    
    1 & Field Background Knowledge 
      & Read and present 1) 
      & Read and present 1) 
      & Read and present 2) 
      & Read and present 2) \\
    \hline
    
    2 & Field Background Knowledge
      & Read and present 3) 
      & Read and present 4) 
      & Read and present 3) 
      & Read and present 4) \\
    \hline
    
    3 & Define the simplest case and prove it 
      & Each proves a lemma (incomplete information)
      & Each proves a lemma (randomness)
      & Each proves a lemma (intersecting paths)
      & Each proves a lemma (duplicate endpoints) \\
    \hline
    
    4 & Consider obstacles
      & Each proves a lemma
      & Each proves a lemma
      & Each proves a lemma
      & Each proves a lemma \\
    \hline
    
    5 & Consider weights on edges
      & Each proves a lemma
      & Each proves a lemma
      & Each proves a lemma
      & Each proves a lemma \\
    \hline
    
    6 & Consider more players
      & Each proves a lemma
      & Each proves a lemma
      & Each proves a lemma
      & Each proves a lemma \\
    \hline
    
    7 & Wrap up
      & Writer
      & Figures
      & Equation
      & Figures \\
    \hline
    
    \end{tabularx}
    \end{table}

The annotation 1) refers to \cite{One}, 2) to \cite{Two}, 3) to \cite{Three}, and 4) to \cite{Four}
    