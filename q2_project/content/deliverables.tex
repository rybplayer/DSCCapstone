\subsection{Project Goals}

Therefore, the goal of our project is to consider relaxations of the above problem statement, and examine the communication complexity for each. Some relaxations we can consider, and their analogy to the application of ``self driving cars'' is as follows:

\begin{itemize}
    \item There are obstacles at nodes, and they are only visible to players who are adjacent to the node of that obstable. This mimics the idea of random roadblocks noticed by the cars.
    \item Edges have weights representing the cost of travelling. The goal is to minimize the total cost of travel. This mimics the idea of some roads being more congested than others.
    \item There are $n$ players. The self-driving cars may form a taxi network.
    \item The games are continuous; every time an endpoint is reached, a new start and end is assigned to each player. This mimics the idea of an taxi service. 
\end{itemize}

\subsection{Actionables}

Therefore, this project can be summarized into three specific, actionables:

\begin{enumerate}
    \item Define new models or problems in the field.
    \item Write a paper that outlines the formal, technical communication complexity bounds for a given multi-agent traversal problem with certain constraints.
    \item Give a poster presentation of the key results of the paper, with the intention of making the results applicable to a wider audience.
\end{enumerate}

Therefore, our work will remain theoretically grounded while remaining applicable for the wider computing community.

\subsection{Deliverables}

We have three main deliverables:

\begin{enumerate}
    \item A paper presenting actionable item 1 and 2.
    \item A research poster of our key results, for actionable 3.
    \item A presentation for the poster, for actionable 3.
\end{enumerate}

To be more specific, the paper will take the form of a research paper, where we define the problems we are working on, prove bounds on these problems, and show how they model situations in the field of multi-agent graph traversal.