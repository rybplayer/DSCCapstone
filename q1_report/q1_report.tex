% !TEX program = lualatex
\documentclass[12pt,letterpaper]{article}
\usepackage[base]{babel}
\usepackage{../resources/style/dsc180reportstyle}

%%%%%%%%%%%%%%%%%%%%%%%%%%%%%%%%%%%%%%%%%%%%%%%%%%%%%%%%
%%%% Title and Authors
%%%%%%%%%%%%%%%%%%%%%%%%%%%%%%%%%%%%%%%%%%%%%%%%%%%%%%%%

\title{DSC Capstone LaTeX Template}

\author{Ryan Batubara \\
  {\tt \small{rbatubara@ucsd.edu}} \\\And
  Ivy Hawks \\
  {\tt \small{mahawks@ucsd.edu}} \\\And
  Darren Ho \\
  {\tt \small{dah103@ucsd.edu}} \\\And
  Ciro Zhang \\
  {\tt \small{ciz001@ucsd.edu}} \\\And
  Shachar Lovett \\
  {\tt \small{slovett@ucsd.edu}} \\}
\normalsize

\begin{document}
\maketitle

%%%%%%%%%%%%%%%%%%%%%%%%%%%%%%%%%%%%%%%%%%%%%%%%%%%%%%%%
%%%% Abstract and Links
%%%%%%%%%%%%%%%%%%%%%%%%%%%%%%%%%%%%%%%%%%%%%%%%%%%%%%%%

\begin{abstract}
    TBA
\end{abstract}

\begin{center}
% Website: \url{https://abc.github.io/} \\
Repository: \url{https://github.com/rybplayer/DSCCapstone}
\end{center}

\maketoc
\clearpage

%%%%%%%%%%%%%%%%%%%%%%%%%%%%%%%%%%%%%%%%%%%%%%%%%%%%%%%%
%%%% Main Contents
%%%%%%%%%%%%%%%%%%%%%%%%%%%%%%%%%%%%%%%%%%%%%%%%%%%%%%%%

\section{Introduction}

\subsection{Background}
Cryptography is the field which deals with secret information. Specifically, it
is used to hide information such that it cannot be intercepted by a third
party. This originally required a symmetric key system, in which all
participating parties used one secret key to decode messages, however modern
systems use Public key encryption schemes, which do not require advance sharing
of a secret key.

\subsection{Foundational Problems}

\subsubsection{Public key Encryption}
A Public key encryption scheme is a system which makes use of trapdoor
functions, functions which can be easily computed but their inversion is
enormously difficult without one key piece of information, such as multiplying
two primes. With one, a public key, or a method to compute the function, can be
shared widely, and any party can pass information through it to generate an
encrypted message. The reverse, however, can only be done (in efficient time)
by the party which created the function, and the message can then only be read
by that party.

\subsubsection{Oblivious Transfer}
An Oblivious transfer is a system in which one party sends some information to
another party, but is not able to determine what, if any information, was sent.
Thus, information can be compared in a function of some kind without the
Senders ability to see what information it was, despite being in possession of
it.

\subsubsection{The Millionaire's Problem}
The Millionaire's Problem, originally proposed by Andrew Yao, is a problem for
efficiently communicating under a public key cryptographic system. Suppose two
Millionaire's, Alice and Bob, wish to know who has more money, but do not wish to
share the exact values. Yao's original solution to this problem involves the
use of a garbled circuit model, in which Alice can generate a circuit which will
compute the solution to the problem, and then encrypt the circuit by decomposing
and anonymizing each step in the process. Alice will compute her portion of
the circuit, encrypt the system such that it can only be decrypted with the
possession of two inputs, allowing bob to complete the circuit using only the
portion of the truth tables which work for his input.

This solution has $\mathcal{O}(n \cdot k)$ communication complexity cost, where $n$
is the size of the input, and $k$ is the security parameter, which in many
modern applications varies from 100 to 128. 

\subsection{Case Study: Garbled Circuits}
Garbled Circuits is one of the most common ways to solve any Oblivious Transfer
protocol, and here we will analyze it's application to the Millionaire's
Problem. 

\subsubsection{Circuit Garbling}
In the problem start, Alice and Bob each have an input  $x, y \in \mathbb{R}^n$. Alice
will create a circuit which computes the function (in this case, Greater Than),
which will take $2 \cdot n$ inputs and produce a Binary output. Alice will then
assign two random strings of length $k$ to each wire in the circuit, where each
string matches to a value 0 or 1 for that wire. Then, using the inputs to each
wire in the circuit, can encrypt the output values for every truth tables. Each
component in the circuit now has a truth table which can only be decrypted with
knowledge of two wire states, and Alice can safely send the random strings
which correspond to her inputs to Bob, who cannot decode what her inputs were
merely from these strings. However, Bob must now acquire the strings which
correspond to his input from Alice, without Alice knowing which inputs she has
sent to Bob.

\subsubsection{Oblivious Transfer Protocol}
Alice will now generate a Public-Private key pair, $t_{pub}$ and $t_{priv}$, as
well as two random messages, $m_0$ and $m_1$, and send both $m_1$ and $m_2$ as
well as $t_{pub}$ to Bob. Bob will chose either $m_1$ or $m_0$, depending on
his input, and generate a random string $k$, and use this to then encrypt $m_b
+ k$, where $b$ is the input string Bob wishes to obtain. Bob can now send this
encrypted message to Alice, who then attempts to decrypt by subtracting both
$m_1$ and $m_2$ from the message bob sent, obtaining two possible options for
$k$, $k_0$ or $k_1$. one of these values will be the original $k$, but Alice
does not know which. She can then use these values to encrypt the random wire
stings and send to Bob, who, knowing the real $k$, can decrypt only the one
corresponding to his chosen input, thereby obtaining only the portion of the
truth table which allows him to compute his input with Alice's.

Given that the communication complexity of the greater than function is
$\omega(n)$, where each decision step in the protocol can be modeled with a
gate or simple circuit, we find that there are then $\omega(n \cdot k)$ total
bits required for this protocol.

\subsection{Modern Applications}
The garbled circuit is not only a useful model for computing two anonymous
values, such as must be done in many e-commerce contexts, where a server must
validate if the purchaser has sufficient funding to make the purchase, but can
be adapted to compute many different functions. As the circuit complexity of
this system can be modeled equivalently to any deterministic two party
communication protocol, it can compute such a protocol anonymously for any
valid function that two parties may wish to compute. This is mainly useful in
transactional contexts, such as with blockchain, where multiple parties are not
behaving with malicious intent, but are nonetheless attempting to violate the
privacy of the other party.


\section{Methods}

\section{Background}
The field of game theory studies the mathematical models of interactions between rational decision makers. Game theory initially addressed the interactions between two parties in a zero sum game but has grown into an umbrella term on rational decision making.

\section{Foundational Problems}

\subsection{Algorithmic Mechanism Design}
Algorithmic mechanism design (AMD) focuses on designing a mechanism where (a) each party behave in a certain way and (b) is computationally efficient. AMD differs from regular mechanism design in that mechanism design assumes infinite compute power. AMD seeks to find a mechanism that performs with certain constraints such as polynomial time.

\subsubsection{Vickrey Auction}
Vickrey Auction is a sealed-bid second-place auction, where the highest bidder pays the second highest price. This is a classical example of AMD because it is in the best interest of each party to bid truthfully.

Say we have three parties participating in this auction, Alice, Bob, and Charlie. Their sealed bids are:
\begin{itemize}
    \item Alice: \$50
    \item Bob: \$100
    \item Charlie: \$70
\end{itemize}

In this scenario, Bob wins and pays \$70.

Bob cannot improve his results by changing his bid, if he raises it the price he pays does not change, and if he lowers it he risks not winning the bid. Hence in this game, there is a dominant strategy to bid truthfully.


\subsection{Nash Equilibrium}
Nash Equilibrium are a foundational problem within game theory that involve finding a state within a scenario where no player can improve their outcome by unilaterally changing their own strategy.

\subsubsection{The Prisoner's Dilemma}
The prisoner's dilemma is one of the simplest examples of finding a \textbf{Nash Equilibrium}. 

Two suspects, Prisoner~A and Prisoner~B, are arrested and interrogated separately. Each prisoner has two possible actions:
\begin{itemize}
    \item \textbf{Cooperate (C)}: remain silent
    \item \textbf{Defect (D)}: betray the other prisoner
\end{itemize}

The payoff matrix (with payoffs in the form $(\text{A's payoff}, \text{B's payoff})$) is given by:

\[
\begin{array}{c|c|c}
    & \text{B: C} & \text{B: D} \\
    \hline
    \text{A: C} & (1, 1) & (3, 0) \\
    \hline
    \text{A: D} & (0, 3) & (2, 2)
\end{array}
\]

\noindent The interpretation is:
\begin{itemize}
    \item If both stay silent (C,C), each receives a light sentence: $1$.
    \item If one defects and the other cooperates, the defector goes free $(0)$ while the cooperator receives the harshest sentence $(3)$.
    \item If both defect (D,D), they each receive a moderate sentence $(2)$.
\end{itemize}

\noindent Even though mutual cooperation leads to a better outcome than mutual defection, the dominant strategy for each prisoner is to defect, making $(D,D)$ the unique Nash equilibrium.

\section{Case Study: Pure Nash Equilibrium}
We will be looking to see what the communication complexity of reaching equilibrium is. More specifically, the communication model in which players initially only know their own utility functions. We want to analyze how much information must be transferred between them to jointly compute the equilibrium point. We also assume that each player follows a predetermine protocol, abstracting any other incentives of each player. We will be focusing on the communication complexity of \emph{pure} Nash Equilibrium. A pure Nash Equilibrium being a situation where there is guaranteed to be a unique best input for a all given parties.

\subsection{Setting for Pure Action Games}
There are $n\ge2$ players, $i=1,2,...,n$. Each player $i$ has a finite set of actions $A_i$ with $|A_i|\ge2$. For the analysis of pure action games, we only consider binary action games, that is, for each $i$, $A_i=\{0,1\}$. Let the joint action space $A=\prod_{i=1}^nA_i$. Each player has a private utility function $u_i:A\xrightarrow{}\{0,1\}^n$ which we are assuming are finitely represented for simplicity.


For a joint action $a=(a_1,...,a_n)\in A$ (for binary action games, a joint action is an n bit binary string), let $a^{-1}$ denote the joint action of all players except player $i$. A joint action $a$ is a \textbf{Pure Nash Equilibrium} if $u_i(a) \ge u_i(b_i, a^{-1})$ for every player $i$ and any action $b_i\in A_i$. That is to say that for all players choosing the alternative option would result in an equal or worse outcome for them.

\subsection{Communication Complexity of Pure Nash Equilibrium}

\textbf{Theorem} \textit{Any pure Nash equilibrium procedure has communication complexity $\Omega(2^n)$}

\textbf{Claim} If there exists a reduction from the S-disjointness problem to an n-person pure Nash Equilibrium procedure that satisfies reducibility and constructibility properties then any pure Nash Equilibrium procedure has communication complexity of at least |S| bits since it has been proven that $CC(DISJ_S) = |S|$.

We want to prove that there exists a reduction from the set disjointness problem where:

\[
S_1\cap S_2 = 0 \text{ or } S_1\cap S_2 \ne 0
\]

Our goal is to relate the disjointness problem into a game that has a pure Nash equilibrium iff the sets intersect.To do so, we use the \textit{matching pennies reduction}.

\textbf{Matching Pennies.}
The matching pennies game is a $2 \times 2$ zero-sum game played between
two players. Each player chooses either \emph{Heads} (H) or \emph{Tails} (T).
If the actions match, player~1 wins; if they mismatch, player~2 wins.
The payoff matrix for player~1 is:
\[
\begin{array}{c|cc}
    & H & T \\ \hline
H & 1 & -1 \\
T & -1 & 1
\end{array}
\]
and player~2 receives the negative of this payoff.

A key property of matching pennies is that it has \emph{no pure Nash
equilibrium}. This makes it useful as a ``destabilizing gadget''
in reductions, since attaching a matching-pennies subgame to an action
profile guarantees that the profile cannot be a pure equilibrium.

For $n \ge 4$, the matching pennies reduction satisfies the
reducibility and constructibility properties.

We first verify the \emph{constructibility} property.
Fix $\alpha \in \{1,2\}$.
By the definition of the matching pennies reduction, the payoff
function $u_{\alpha,i}(a)$ of every player $(\alpha,i)$ depends only on
the following information:

\begin{itemize}
    \item whether $a \in S_\alpha$, and 
    \item the actions $a_{h,1}$ and $a_{h,2}$ in case $a \notin S_h$
\end{itemize}

Since $S_\alpha$ is the private input of the agents on side $\alpha$ in
the disjointness problem, the event $a \in S_\alpha$ is computable by
those agents, and $a_{\alpha,1},a_{\alpha,2}$ are part of the observed
joint action $a$. Hence each payoff $u_{\alpha,i}(a)$ is computable
from $(a,S_\alpha,i)$ alone. This establishes constructibility.

We now prove the \emph{reducibility} property.
We must show that
\[
S_1 \cap S_2 \neq \emptyset 
\qquad\Longleftrightarrow\qquad
G \text{ has a pure Nash equilibrium}.
\]

Suppose $a \in S_1 \cap S_2$.
Then by construction, every player in $T_1$ and $T_2$ receives payoff
$2$ at $a$, which is the maximal payoff any player can obtain.
Thus no player can improve by deviating, and $a$ is a pure Nash
equilibrium.

Suppose $a \notin S_\alpha$ for some $\alpha \in \{1,2\}$.
Then the two designated players $(\alpha,1)$ and $(\alpha,2)$ play a
matching pennies game at $a$.
The matching pennies game has no pure Nash equilibrium: at every pure
action profile of the two players, one of them strictly benefits from
unilaterally deviating.
Therefore, at $a$ at least one of the players $(\alpha,1)$ or
$(\alpha,2)$ has a profitable deviation, implying that $a$ cannot be a
pure Nash equilibrium of $G$.
Hence no pure Nash equilibrium can lie outside $S_1 \cap S_2$.

Combining the two directions, we conclude that a pure Nash equilibrium
exists in $G$ if and only if $S_1 \cap S_2 \neq \emptyset$.

Since we have proven the claim, we know that the communication complexity of any pure Nash Equilibrium is $CC(DISJ_s)=|S|$. Remember that $S=\{0,1\}^n$ so therefore $|S|=2^n$ proving the theorem.

\section{Broad Applications}

The communication complexity of Game Theory has many varied applications and is prevalent in any situation that involves optimizing the output in any multiparty game. Game Theory draws naturally has parallels to other fields such as multi-agent and distributed systems.

\section{Results}

\section{Discussion}

\section{Conclusion}

\section{\LaTeX{} Typesetting Examples}

{\color{blue} This is not a real section; it's just here to show examples of how to format various components. Remove it before submitting!}

\subsection{\LaTeX{} Basics}

Here's \textbf{bold} and \textit{italicized} text. Here's \texttt{text\_that\_looks.like(code)}.

\begin{itemize}
    \item Here's a regular bulleted list item.
    \item And another.
\end{itemize}

Here's a \href{https://datascience.ucsd.edu}{hyperlink}. If you want to use a numbered list, you can experiment with:

\begin{enumerate}
    \item This.
    \item This.
    \item And this.
\end{enumerate}

Here's how you might include a snippet of actual code:

\begin{verbatim}
# If you want to use syntax highlighting, look into the minted package.
def f(x):
    return 2 * x + 3
\end{verbatim}

Here's how you might format a single equation:

$$\int_{-\infty}^\infty f_X(x)dx = 1$$

And a chain of equations:

\begin{align*}
    \frac{1}{n}\sum_{i = 1}^n (x_i - \bar{x})^2 &= \frac{1}{n}\sum_{i = 1}^n (x_i^2 - 2x_i\bar{x} + \bar{x}^2)
    \\ &= \frac{1}{n}\sum_{i = 1}^n x_i^2 - \frac{2}{n}\bar{x}\sum_{i = 1}^n x_i + \frac{\bar{x}^2}{n}\sum_{i = 1}^n 1
    \\ &= \frac{1}{n}\sum_{i = 1}^n x_i^2 - 2\bar{x}^2 + \bar{x}^2
    \\ &= \frac{1}{n}\sum_{i = 1}^n x_i^2 - \bar{x}^2
\end{align*}


\subsection{Figure Examples}

Here are some example figures. 
Figure \ref{fig:somefig1} presents a scatter plot.

\begin{figure}[htbp]
\centering
% \includegraphics[width=.65\linewidth]{figure/somefig1.pdf}
\caption{Yes, put a few words or sentences here explaining what is in the figure.}
\label{fig:somefig1}
\end{figure}

Figure \ref{fig:someotherfigs} presents some summaries of the performance of our model.
The left panel of Figure \ref{fig:someotherfigs} presents something.
The right panel of Figure \ref{fig:someotherfigs} presents some other things.

\begin{figure}[htbp]
\begin{minipage}{0.53\linewidth}
  \centering
%   \includegraphics[width=\linewidth]{figure/somefig2.png}
\end{minipage}
\begin{minipage}{0.42\linewidth}
  \centering
%   \includegraphics[width=\linewidth]{figure/somefig3.png}
\end{minipage}
\caption{You can put figures side-by-side as well.}
\label{fig:someotherfigs}
\end{figure}


\subsection{Table Examples}

Table \ref{tab:sometab1} presents some summary of the data.

\begin{table}[htbp]
\caption{Some Table Caption}
\label{tab:sometab1}
% \resizebox{0.4\linewidth}{!}{\input{table/sometab1}}
\end{table}

Table \ref{tab:sometab2} presents some summaries of the performance of our model.

\begin{table}[htbp]
\caption{Some Other Table Caption}
\label{tab:sometab2}
% \resizebox{0.9\linewidth}{!}{\input{table/sometab2}}
\end{table}

\subsection{Equations and Algorithms Examples}

Algorithm \ref{alg:fuzzyKmeans} implements Fuzzy K-means.

\begin{algorithm}
\caption{Fuzzy K-means clustering algorithm}
\label{alg:fuzzyKmeans}
\begin{enumerate}
    \item Choose primary centroids $v_{k}$
    \item Compute the membership degree of all feature vectors in all clusters
    \begin{equation}
    u_{ki}  = \frac{1}{ \sum_{j=1}^K ( \frac{D^{2}(x_{i} - v_{k})}{D^{2}(x_{i} - v_{j})})^\frac{2} 
    {m-1}}
    \label{eq:kmeans}
    \end{equation}
\end{enumerate}
\end{algorithm}

Algorithm \ref{alg:net} calculates net activation.


\begin{algorithm}
\caption{Computing Net Activation}
\label{alg:net}
% \DontPrintSemicolon
% \LinesNumbered
\KwIn{$x_1, \ldots, x_n, w_1, \ldots, w_n$}
\KwOut{$y$, the net activation}
$y\leftarrow 0$\;
\For{$i\leftarrow 1$ \KwTo $n$}{
$y \leftarrow y + w_i*x_i$\;
}
\end{algorithm}

In Variational Autoencoder (VAE), we directly maximize the Evidence Lower Bound (ELBO) using the following Equations \ref{eq:bla}--\ref{eq:blablabla}.
\begin{align}
  \mathbb{E}_{q_{\boldsymbol{\phi}}(\boldsymbol{z}\mid\boldsymbol{x})}\left[\log\frac{p(\boldsymbol{x}, \boldsymbol{z})}{q_{\boldsymbol{\phi}}(\boldsymbol{z}\mid\boldsymbol{x})}\right]
  &= \mathbb{E}_{q_{\boldsymbol{\phi}}(\boldsymbol{z}\mid\boldsymbol{x})}\left[\log\frac{p_{\boldsymbol{\theta}}(\boldsymbol{x}\mid\boldsymbol{z})p(\boldsymbol{z})}{q_{\boldsymbol{\phi}}(\boldsymbol{z}\mid\boldsymbol{x})}\right] \label{eq:bla} \\
  &= \mathbb{E}_{q_{\boldsymbol{\phi}}(\boldsymbol{z}\mid\boldsymbol{x})}\left[\log p_{\boldsymbol{\theta}}(\boldsymbol{x}\mid\boldsymbol{z})\right] + \mathbb{E}_{q_{\boldsymbol{\phi}}(\boldsymbol{z}\mid\boldsymbol{x})}\left[\log\frac{p(\boldsymbol{z})}{q_{\boldsymbol{\phi}}(\boldsymbol{z}\mid\boldsymbol{x})}\right] \label{eq:blabla} \\
  &= \underbrace{\mathbb{E}_{q_{\boldsymbol{\phi}}(\boldsymbol{z}\mid\boldsymbol{x})}\left[\log p_{\boldsymbol{\theta}}(\boldsymbol{x}\mid\boldsymbol{z})\right]}_\text{reconstruction term} - \underbrace{\mathcal{D}_{\text{KL}}(q_{\boldsymbol{\phi}}(\boldsymbol{z}\mid\boldsymbol{x}) \mid\mid p(\boldsymbol{z}))}_\text{prior matching term} \label{eq:blablabla}
\end{align}

\subsection{Inline Citation Examples}

Citation in text (no parentheses): use \texttt{{\textbackslash}cite\{citekey\}}. 
For example, \cite{breiman2011}, \cite{devlin2019bert}.

Citation in parentheses: use \texttt{{\textbackslash}citep\{citekey\}}. 
For example: \citep{vaswani2023attention}, \citep{karras2019stylebased}.


%%%%%%%%%%%%%%%%%%%%%%%%%%%%%%%%%%%%%%%%%%%%%%%%%%%%%%%%
%%%% Reference / Bibliography
%%%%%%%%%%%%%%%%%%%%%%%%%%%%%%%%%%%%%%%%%%%%%%%%%%%%%%%%

\makereference

{\color{blue} To edit the contents of the ``References" section, edit \texttt{reference.bib}. Many conference websites format citations in BibTeX that you can copy into \texttt{reference.bib} directly; you can also search for the paper on Google Scholar, click ``Cite", and then click ``BibTeX" (\href{https://scholar.google.com/scholar?hl=en&as_sdt=0%2C23&q=attention+is+all+you+need&btnG=#d=gs_cit&t=1700436667623&u=%2Fscholar%3Fq%3Dinfo%3A5Gohgn6QFikJ%3Ascholar.google.com%2F%26output%3Dcite%26scirp%3D0%26hl%3Den}{here}'s an example).}

\bibliography{reference}
\bibliographystyle{style/dsc180bibstyle}

%%%%%%%%%%%%%%%%%%%%%%%%%%%%%%%%%%%%%%%%%%%%%%%%%%%%%%%%
%%%% Appendix
%%%%%%%%%%%%%%%%%%%%%%%%%%%%%%%%%%%%%%%%%%%%%%%%%%%%%%%%

\clearpage
\makeappendix

\subsection{Training Details}

\subsection{Additional Figures}

\subsection{Additional Tables}


\end{document}
