Communication complexity is the study of asymptotic communication bounds. For example, suppose two parties commonly named Alice and Bob each have a set of $n$ numbers, and they wish to find if the intersection of their sets are empty. Suppose further that Alice and Bob each have access to infinite computational power. That is, we only care about the communication that happens between them.

Even with such a powerful model, communication complexity shows that such an operation must still take $\Omega(n)$ bits. Despite the power of our assumptions and the simplicity of the model, disjointness can be used to prove lower bounds for many models. For example, disjointness shows that any Turing machine with one tape must take at least $\Omega(n^2)$ time to compute disjointness written on its tape. We show this by means of a reduction.

Suppose such a machine takes $T < \Omega(n^2)$ time to compute disjointness of two sets written on a tape. Alice and Bob use it to solve the disjointness problem in less than $\Omega(n)$ communication as follows: Both Alice and Bob run the machine. Whenever the machine needs to read from Alice's part of the tape, Alice sends Bob a message to indicate what needs to be executed next. The tape can only cross between Alice and Bob's part of the tape, or vice versa, at most $T/n$ times, because each sweep will take $\Omega(n)$ time. Therefore, using this protocol, Alice and Bob obtain a method for computing disjointess in less than $\Omega(n)$ time, a contradiction.

In this report, we survey how foundational problems in communication complexity, such as disjointness, can be used to prove bounds on a variety of problems from different fields. Four areas have become our focus: cryptography, game theory, machine learning, and agential learning. For these areas, we begin by reviewing the foundational problems in the field, before jumping into the technical proof details of bounds for such problems. We conclude by providing a cursory overview of other application areas.