\subsection{Background}
Cryptography is the field which deals with secret information. Specifically, it
is used to hide information such that it cannot be intercepted by a third
party. This originally required a symmetric key system, in which all
participating parties used one secret key to decode messages, however modern
systems use Public key encryption schemes, which do not require advance sharing
of a secret key.

\subsection{Foundational Problems}

\subsubsection{Public key Encryption}
A Public key encryption scheme is a system which makes use of trapdoor
functions, functions which can be easily computed but their inversion is
enormously difficult without one key piece of information, such as multiplying
two primes. With one, a public key, or a method to compute the function, can be
shared widely, and any party can pass information through it to generate an
encrypted message. The reverse, however, can only be done (in efficient time)
by the party which created the function, and the message can then only be read
by that party.

\subsubsection{Oblivious Transfer}
An Oblivious transfer is a system in which one party sends some information to
another party, but is not able to determine what, if any information, was sent.
Thus, information can be compared in a function of some kind without the
Senders ability to see what information it was, despite being in possession of
it.

\subsubsection{The Millionaire's Problem}
The Millionaire's Problem, originally proposed by Andrew Yao, is a problem for
efficiently communicating under a public key cryptographic system. Suppose two
Millionaire's, Alice and Bob, wish to know who has more money, but do not wish to
share the exact values. Yao's original solution to this problem involves the
use of a garbled circuit model, in which Alice can generate a circuit which will
compute the solution to the problem, and then encrypt the circuit by decomposing
and anonymizing each step in the process. Alice will compute her portion of
the circuit, encrypt the system such that it can only be decrypted with the
possession of two inputs, allowing bob to complete the circuit using only the
portion of the truth tables which work for his input.

This solution has $\mathcal{O}(n \cdot k)$ communication complexity cost, where $n$
is the size of the input, and $k$ is the security parameter, which in many
modern applications varies from 100 to 128. 

\subsection{Case Study: Garbled Circuits}
Garbled Circuits is one of the most common ways to solve any Oblivious Transfer
protocol, and here we will analyze it's application to the Millionaire's
Problem. 

\subsubsection{Circuit Garbling}
In the problem start, Alice and Bob each have an input  $x, y \in \mathbb{R}^n$. Alice
will create a circuit which computes the function (in this case, Greater Than),
which will take $2 \cdot n$ inputs and produce a Binary output. Alice will then
assign two random strings of length $k$ to each wire in the circuit, where each
string matches to a value 0 or 1 for that wire. Then, using the inputs to each
wire in the circuit, can encrypt the output values for every truth tables. Each
component in the circuit now has a truth table which can only be decrypted with
knowledge of two wire states, and Alice can safely send the random strings
which correspond to her inputs to Bob, who cannot decode what her inputs were
merely from these strings. However, Bob must now acquire the strings which
correspond to his input from Alice, without Alice knowing which inputs she has
sent to Bob.

\subsubsection{Oblivious Transfer Protocol}
Alice will now generate a Public-Private key pair, $t_{pub}$ and $t_{priv}$, as
well as two random messages, $m_0$ and $m_1$, and send both $m_1$ and $m_2$ as
well as $t_{pub}$ to Bob. Bob will chose either $m_1$ or $m_0$, depending on
his input, and generate a random string $k$, and use this to then encrypt $m_b
+ k$, where $b$ is the input string Bob wishes to obtain. Bob can now send this
encrypted message to Alice, who then attempts to decrypt by subtracting both
$m_1$ and $m_2$ from the message bob sent, obtaining two possible options for
$k$, $k_0$ or $k_1$. one of these values will be the original $k$, but Alice
does not know which. She can then use these values to encrypt the random wire
stings and send to Bob, who, knowing the real $k$, can decrypt only the one
corresponding to his chosen input, thereby obtaining only the portion of the
truth table which allows him to compute his input with Alice's.

Given that the communication complexity of the greater than function is
$\omega(n)$, where each decision step in the protocol can be modeled with a
gate or simple circuit, we find that there are then $\omega(n \cdot k)$ total
bits required for this protocol.

\subsection{Modern Applications}
The garbled circuit is not only a useful model for computing two anonymous
values, such as must be done in many e-commerce contexts, where a server must
validate if the purchaser has sufficient funding to make the purchase, but can
be adapted to compute many different functions. As the circuit complexity of
this system can be modeled equivalently to any deterministic two party
communication protocol, it can compute such a protocol anonymously for any
valid function that two parties may wish to compute. This is mainly useful in
transactional contexts, such as with blockchain, where multiple parties are not
behaving with malicious intent, but are nonetheless attempting to violate the
privacy of the other party.
