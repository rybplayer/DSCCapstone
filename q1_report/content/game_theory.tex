\section{Background}
The field of game theory studies the mathematical models of interactions between rational decision makers. Game theory initially addressed the interactions between two parties in a zero sum game but has grown into an umbrella term on rational decision making.

\section{Foundational Problems}

\subsection{Algorithmic Mechanism Design}
Algorithmic mechanism design (AMD) focuses on designing a mechanism where (a) each party behave in a certain way and (b) is computationally efficient. AMD differs from regular mechanism design in that mechanism design assumes infinite compute power. AMD seeks to find a mechanism that performs with certain constraints such as polynomial time.

\subsubsection{Vickrey Auction}
Vickrey Auction is a sealed-bid second-place auction, where the highest bidder pays the second highest price. This is a classical example of AMD because it is in the best interest of each party to bid truthfully.

Say we have three parties participating in this auction, Alice, Bob, and Charlie. Their sealed bids are:
\begin{itemize}
    \item Alice: \$50
    \item Bob: \$100
    \item Charlie: \$70
\end{itemize}

In this scenario, Bob wins and pays \$70.

Bob cannot improve his results by changing his bid, if he raises it the price he pays does not change, and if he lowers it he risks not winning the bid. Hence in this game, there is a dominant strategy to bid truthfully.


\subsection{Nash Equilibrium}
Nash Equilibrium are a foundational problem within game theory that involve finding a state within a scenario where no player can improve their outcome by unilaterally changing their own strategy.

\subsubsection{The Prisoner's Dilemma}
The prisoner's dilemma is one of the simplest examples of finding a \textbf{Nash Equilibrium}. 

Two suspects, Prisoner~A and Prisoner~B, are arrested and interrogated separately. Each prisoner has two possible actions:
\begin{itemize}
    \item \textbf{Cooperate (C)}: remain silent
    \item \textbf{Defect (D)}: betray the other prisoner
\end{itemize}

The payoff matrix (with payoffs in the form $(\text{A's payoff}, \text{B's payoff})$) is given by:

\[
\begin{array}{c|c|c}
    & \text{B: C} & \text{B: D} \\
    \hline
    \text{A: C} & (1, 1) & (3, 0) \\
    \hline
    \text{A: D} & (0, 3) & (2, 2)
\end{array}
\]

\noindent The interpretation is:
\begin{itemize}
    \item If both stay silent (C,C), each receives a light sentence: $1$.
    \item If one defects and the other cooperates, the defector goes free $(0)$ while the cooperator receives the harshest sentence $(3)$.
    \item If both defect (D,D), they each receive a moderate sentence $(2)$.
\end{itemize}

\noindent Even though mutual cooperation leads to a better outcome than mutual defection, the dominant strategy for each prisoner is to defect, making $(D,D)$ the unique Nash equilibrium.

\section{Case Study: Pure Nash Equilibrium}
We will be looking to see what the communication complexity of reaching equilibrium is. More specifically, the communication model in which players initially only know their own utility functions. We want to analyze how much information must be transferred between them to jointly compute the equilibrium point. We also assume that each player follows a predetermine protocol, abstracting any other incentives of each player. We will be focusing on the communication complexity of \emph{pure} Nash Equilibrium. A pure Nash Equilibrium being a situation where there is guaranteed to be a unique best input for a all given parties.

\subsection{Setting for Pure Action Games}
There are $n\ge2$ players, $i=1,2,...,n$. Each player $i$ has a finite set of actions $A_i$ with $|A_i|\ge2$. For the analysis of pure action games, we only consider binary action games, that is, for each $i$, $A_i=\{0,1\}$. Let the joint action space $A=\prod_{i=1}^nA_i$. Each player has a private utility function $u_i:A\xrightarrow{}\{0,1\}^n$ which we are assuming are finitely represented for simplicity.


For a joint action $a=(a_1,...,a_n)\in A$ (for binary action games, a joint action is an n bit binary string), let $a^{-1}$ denote the joint action of all players except player $i$. A joint action $a$ is a \textbf{Pure Nash Equilibrium} if $u_i(a) \ge u_i(b_i, a^{-1})$ for every player $i$ and any action $b_i\in A_i$. That is to say that for all players choosing the alternative option would result in an equal or worse outcome for them.

\subsection{Communication Complexity of Pure Nash Equilibrium}

\textbf{Theorem} \textit{Any pure Nash equilibrium procedure has communication complexity $\Omega(2^n)$}

\textbf{Claim} If there exists a reduction from the S-disjointness problem to an n-person pure Nash Equilibrium procedure that satisfies reducibility and constructibility properties then any pure Nash Equilibrium procedure has communication complexity of at least |S| bits since it has been proven that $CC(DISJ_S) = |S|$.

We want to prove that there exists a reduction from the set disjointness problem where:

\[
S_1\cap S_2 = 0 \text{ or } S_1\cap S_2 \ne 0
\]

Our goal is to relate the disjointness problem into a game that has a pure Nash equilibrium iff the sets intersect.To do so, we use the \textit{matching pennies reduction}.

\textbf{Matching Pennies.}
The matching pennies game is a $2 \times 2$ zero-sum game played between
two players. Each player chooses either \emph{Heads} (H) or \emph{Tails} (T).
If the actions match, player~1 wins; if they mismatch, player~2 wins.
The payoff matrix for player~1 is:
\[
\begin{array}{c|cc}
    & H & T \\ \hline
H & 1 & -1 \\
T & -1 & 1
\end{array}
\]
and player~2 receives the negative of this payoff.

A key property of matching pennies is that it has \emph{no pure Nash
equilibrium}. This makes it useful as a ``destabilizing gadget''
in reductions, since attaching a matching-pennies subgame to an action
profile guarantees that the profile cannot be a pure equilibrium.

For $n \ge 4$, the matching pennies reduction satisfies the
reducibility and constructibility properties.

We first verify the \emph{constructibility} property.
Fix $\alpha \in \{1,2\}$.
By the definition of the matching pennies reduction, the payoff
function $u_{\alpha,i}(a)$ of every player $(\alpha,i)$ depends only on
the following information:

\begin{itemize}
    \item whether $a \in S_\alpha$, and 
    \item the actions $a_{h,1}$ and $a_{h,2}$ in case $a \notin S_h$
\end{itemize}

Since $S_\alpha$ is the private input of the agents on side $\alpha$ in
the disjointness problem, the event $a \in S_\alpha$ is computable by
those agents, and $a_{\alpha,1},a_{\alpha,2}$ are part of the observed
joint action $a$. Hence each payoff $u_{\alpha,i}(a)$ is computable
from $(a,S_\alpha,i)$ alone. This establishes constructibility.

We now prove the \emph{reducibility} property.
We must show that
\[
S_1 \cap S_2 \neq \emptyset 
\qquad\Longleftrightarrow\qquad
G \text{ has a pure Nash equilibrium}.
\]

Suppose $a \in S_1 \cap S_2$.
Then by construction, every player in $T_1$ and $T_2$ receives payoff
$2$ at $a$, which is the maximal payoff any player can obtain.
Thus no player can improve by deviating, and $a$ is a pure Nash
equilibrium.

Suppose $a \notin S_\alpha$ for some $\alpha \in \{1,2\}$.
Then the two designated players $(\alpha,1)$ and $(\alpha,2)$ play a
matching pennies game at $a$.
The matching pennies game has no pure Nash equilibrium: at every pure
action profile of the two players, one of them strictly benefits from
unilaterally deviating.
Therefore, at $a$ at least one of the players $(\alpha,1)$ or
$(\alpha,2)$ has a profitable deviation, implying that $a$ cannot be a
pure Nash equilibrium of $G$.
Hence no pure Nash equilibrium can lie outside $S_1 \cap S_2$.

Combining the two directions, we conclude that a pure Nash equilibrium
exists in $G$ if and only if $S_1 \cap S_2 \neq \emptyset$.

Since we have proven the claim, we know that the communication complexity of any pure Nash Equilibrium is $CC(DISJ_s)=|S|$. Remember that $S=\{0,1\}^n$ so therefore $|S|=2^n$ proving the theorem.

\section{Broad Applications}

The communication complexity of Game Theory has many varied applications and is prevalent in any situation that involves optimizing the output in any multiparty game. Game Theory draws naturally has parallels to other fields such as multi-agent and distributed systems.