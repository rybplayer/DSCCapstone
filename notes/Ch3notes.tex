\documentclass{article}

\usepackage[a4paper, total={6in, 8in}]{geometry}

\author{Ciro, Darren, Ivy, Ryan}
\title{DSC180A FA25 Notes}
\date{\today}

\begin{document}
\tableofcontents

\section{Randomized Protocals}
\subsection{Types of Randomness}
A \emph{Randomized} Protocal is a protocal which has access to some form of
randomness. There are three veriaties of randomness that can occur:

\begin{description}
\item[Public Coin]
A Protocal whos randomness is derived from a string, known as a \emph{coin},
which is visible to all parties.
\item[Provate coin]
Much like public coin, only in this case each party has a uniqe coin to produce
their own randomness
\item[Randomized Input]
In this case, we considter that the input is randomly sampled from a
distribution, which has the effect of producing a non determanistic protocal
\end{description}

For any random protocal, we can convert it easily into a determanistic
protocal, simply by fixing a coin or input, and then consiter how it behaves.
Further, Public and Private coin protocals can be cycled between. When
converting from a private to a public protocal, we mearely need to had both
parties agree to share thier private coins. For public to private, it can be
shown that this conversion can be done with only an additional
$\log(\frac{n}{\epsilon^2}) + \Omicron(1)$ bits.
\subsection{Error}
The overall error rate of a Randomized protocal, $\epsilon$, is thought of as
the worst case error. That is, for all inputs $(x, y)$, the error rate of the
protocal is never more that $\epsilon$. Furthermore, as we assume this error is
less than $\frac{1}{2}$, it can always be reduced by rerunning the protocal. To
reduce the error to any given value, $\delta$, the protocal can be repeated
$\Omicron(\log(\frac{1}{\delta})$ times.

\subsection{Yaos minimax Principle}
Yao's Minimax principle is a very helpful principle for proving lowerbounds on
randomized protocals. Formaly, it is an extention of the general minimax
principle, $\min_{x\inA} \max_{y \in b} xMy = \max_{y \in B} \min_{x \in A}$,
to expectations over random variables. Informally, it states that the optimal
performance of any random protocal is equal to that of any determanistic
protocal who's input is sampled from a worst case distribution, chosen to be as
hard as possible for the algorithm to solve. Thus, for many random protocals,
the process of finding a lower bound becomes that of finding such a worste case
input, and then proving the lower bound for that input.
\end{document}
